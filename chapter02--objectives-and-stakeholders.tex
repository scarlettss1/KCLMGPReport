\chapter{Objectives and stakeholders}
\label{sect:objectives}

\defaultInstructions

\begin{length}
There are no strict length requirements for this chapter.  However, it is important that the writing is clear and concise.  Avoid repeating yourself.  It is expected that a typical project needs 1--3 pages, but this can vary considerably from project to project.
\end{length}

\begin{expectations}
This is a short chapter that identifies the objectives and stakeholders of the project.  A good report also discusses how the different types of stakeholder are affected by the project or how they could be affected by the project.

As you explain the project objectives, focus exclusively on what value the system adds to the stakeholders.  Your objectives should justify \emph{why} you are building the system.  The software requirements are \emph{not} objectives.  The software requirements explain \emph{what} your system can do.  In other words, they are a way to achieve objectives.  Software requirements do \emph{not} belong in this chapter.  Your project should have at least one objective.  Otherwise, the project is a pointless exercise.  Many objectives do not necessarily make for a better project.

Some stakeholders of a project will be obvious: e.g. the people who the beneficiaries of the value your project adds.  However, it is important to be aware of all stakeholders.  The stakeholder D.A.N.C.E. tool to identify all stakeholders and it is explained in a video on project initiation.  Consider carefully how each stakeholder is affected by your project and who the key stakeholders are.

The section headers below provide a suggested structure.  You should feel free to change it or extend it.
\end{expectations}

\section{Project objectives}
\label{sect:objectives}

\section{Stakeholder}
\label{sect:stakeholder}

\subsection{Stakeholder analysis}
\subsection{Key stakeholders}