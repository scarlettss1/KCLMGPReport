\chapter{Objectives and stakeholders}
\label{sect:objectives}

\defaultInstructions

\begin{length}
There are no strict length requirements for this chapter.  However, it is important that the writing is clear and concise.  Avoid repeating yourself.  It is expected that a typical project needs 1--3 pages, but this can vary considerably from project to project.
\end{length}

\begin{expectations}
This is a short chapter that identifies the objectives and stakeholders of the project.  A good report also discusses how the different types of stakeholder are affected by the project or how they could be affected by the project.

As you explain the project objectives, focus exclusively on what value the system adds to the stakeholders.  Your objectives should justify \emph{why} you are building the system.  The software requirements are \emph{not} objectives.  The software requirements explain \emph{what} your system can do.  In other words, they are a way to achieve objectives.  Software requirements do \emph{not} belong in this chapter.  Your project should have at least one objective.  Otherwise, the project is a pointless exercise.  Many objectives do not necessarily make for a better project.

Some stakeholders of a project will be obvious: e.g. the people who the beneficiaries of the value your project adds.  However, it is important to be aware of all stakeholders.  The stakeholder D.A.N.C.E. tool to identify all stakeholders and it is explained in a video on project initiation.  Consider carefully how each stakeholder is affected by your project and who the key stakeholders are.

The section headers below provide a suggested structure.  You should feel free to change it or extend it.
\end{expectations}

\section{Project objectives}
\label{sect:objectives}
The main objective of this project is to create a flexible, user-friendly habit and activity tracking app that helps people monitor and improve their productivity in a simple, yet customisable way. 

The value our system provides to the stakeholders includes:

\begin{itemize}
    \item \textbf{Better Productivity:} Users can easily track their habits and activities without too much manual input, helping them stay on top of their goals.
    
    \item \textbf{Smart Tracking:} Using location data, mobile activity, app usage, and AI, the app will automatically recognise what users are doing based on the information provided.

    \item \textbf{Customisable Habit Tracking:} Users can define habits in their own way, making the system more flexible and suited to different productivity styles. Most habit trackers enforce strict schedules, marking habits as "failed" if missed. Our app offers a flexible approach, letting users set goals like "exercise 3 times a week," dynamically adjusting streaks and sending only a weekly reminder to reduce pressure.

    \item \textbf{Privacy and Control:} Users will have full control over their data, with options to grant permissions, delete data, and choose which apps are tracked.

    \item \textbf{Data Insights:} The app will offer user-friendly, customisable graphs and reports to help users analyse their progress and export this data if needed. This allows users to assess their progress over a period of time, which can be a powerful tool to see where improvements towards their goals can be made in the future.
\end{itemize}

\section{Stakeholder}
\label{sect:stakeholder}

\subsection{Stakeholder analysis}



\subsection{Key stakeholders}