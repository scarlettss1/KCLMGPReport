\chapter{Specifications}
\label{chap:specifications}

\defaultInstructions

\begin{length}
There are no strict length requirements for this chapter.  However, it is important that the writing is clear and concise.  Avoid repeating yourself.  It is expected that a typical project needs 1--5 pages, but this can vary considerably from project to project.
\end{length}

\begin{expectations}
This chapter should provide the specifications of the software that you have implemented.  Good specifications are clear, concise, consistent (with one another and with the project objectives), estimable (the effort required of functional requirements should be estimable), and testable (you should be able to create automated or manual tests for these specifications).

A significant part of this chapter covers functional requirements/specifications.  These refer to the product features your team has developed.  Present a clear overview of all functional requirements/specifications that can be found in the deployed system.  For each feature, explain how it contributes to the project objectives, and explain how/where it can be found.  Do \emph{not} mix the specifications the team managed to implement with those they did not.  

If necessary or helpful for clarity, explain what is out of scope of the project in a distinct section.  You may discuss ideas for functional specifications the team did not manage to implement, provided this discussion is clearly distinct from the project's achieved functional specifications.

The project objectives will also entail certain non-functional specifications.  These may include, for example, reliability, usability, portability, scalability, performance, compatibility, security, compliance, etc.  Identify which non-functional specifications are important to this project and define them.  Make sure to present them in order of importance, and relate each one to the project objectives.  

The section headers below provide a suggested structure.  You should feel free to change it or extend it.
\end{expectations}

\section{Functional specifications}
\label{sect:functional-specifications}

\section{Non-functional specifications}
\label{sect:non-functional-specifications}
