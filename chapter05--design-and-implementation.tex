\chapter{Design and implementation}
\label{chap:design-and-implementation}

\defaultInstructions

\begin{length}
There are no strict length requirements for this chapter.  However, it is important that the writing is clear and concise.  Avoid repeating yourself.  Make sure to clearly signpost evidence for claims.  It should be possible for a reader to understand this chapter without consulting other sources.  It is expected that a typical project needs 3--6 pages, but this can vary considerably from project to project.
\end{length}

\begin{expectations}
In this chapter, the team should discuss the key design and implementation decisions made during the project, and how these are reflected in the end product.  Design refers to the way the software system is organised into smaller components and the way in which they are related.  We are \emph{not} looking for a discussion of a series of screenshots of your application.

Normally, all teams should present the overall architecture of the software.  The remainder of the chapter will depend on the type of system that is built.  To decide what to cover, consider the following to aspects of the project:
\begin{itemize}
\item What design and implementation decisions did you make to improve software quality?
\item What design and implementation decisions did you make to achieve challenging functional and non-functional specifications
\end{itemize}
Depending on your answers to these questions, you may cover the design an implementation specific components, interfaces, the overall class structure, the database design, algorithms, business processes, etc.

A good report considers alternative options for key decisions and includes sound justifications for the decisions made.  If applicable, you may reflect on changes made during the project.  Ideally, key decisions are rooted in Software Engineering theory.
\end{expectations}

\section{Architecture}
\label{sect:architecture}
\begin{expectations}
Add other sections and sub-sections as necessary.
\end{expectations}